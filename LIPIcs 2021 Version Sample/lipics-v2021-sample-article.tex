
\documentclass[
  a4paper,
  oneside,         % ← 單面模式,取消左右邊對稱排版
  UKenglish,       % ← 英式英文斷字規則
  cleveref,        % ← 使用 cleveref 自動引用
  autoref,         % ← 使用 \autoref{...}
  thm-restate      % ← 支援 Theorem restate
]{lipics-v2021}

%─────────────────────────────────────────────────────────────────────────
%  (1) 先擋掉 lipics-v2021 裡面預設開的所有行號機制
%      \nolinenumbers 只能 “不印出行號” ,但還是會保留行號的那段額外 margin。
%      因此下面兩行要確保「不只不顯示」,而是「完全把 lineno 套件掐死」:
%─────────────────────────────────────────────────────────────────────────

\nolinenumbers
\let\makeLineNumber\relax     % 關閉 makeLineNumber(lineno 內部指令)
\let\thelinenumber\relax      % 關閉 thelinenumber
\let\linenumberfont\relax     % 關閉 linenumberfont
\let\linenumbers\relax        % 關閉 \linenumbers 指令
\let\endlineumbers\relax      % 關閉 \endlineumbers
\let\syncfootnotelinenumbers\relax
\let\resetlinenumberwidth\relax
\let\setrunninglinenumbers\relax
\let\setpagewiselinenumbers\relax
%This is a template for producing LIPIcs articles. 
%See lipics-v2021-authors-guidelines.pdf for further information.
%for A4 paper format use option "a4paper", for US-letter use option "letterpaper"
%for british hyphenation rules use option "UKenglish", for american hyphenation rules use option "USenglish"
%for section-numbered lemmas etc., use "numberwithinsect"
%for enabling cleveref support, use "cleveref"
%for enabling autoref support, use "autoref"
%for anonymousing the authors (e.g. for double-blind review), add "anonymous"
%for enabling thm-restate support, use "thm-restate"
%for enabling a two-column layout for the author/affilation part (only applicable for > 6 authors), use "authorcolumns"
%for producing a PDF according the PDF/A standard, add "pdfa"

%\pdfoutput=1 %uncomment to ensure pdflatex processing (mandatatory e.g. to submit to arXiv)
%\hideLIPIcs  %uncomment to remove references to LIPIcs series (logo, DOI, ...), e.g. when preparing a pre-final version to be uploaded to arXiv or another public repository

%\graphicspath{{./graphics/}}%helpful if your graphic files are in another directory

\bibliographystyle{plainurl}% the mandatory bibstyle

\title{A Modular Platform for Attack and Defense of Customizable Text-based CAPTCHAs}
\pagestyle{empty}
%\titlerunning{Dummy short title} %TODO optional, please use if title is longer than one line
\usepackage{fontspec}
\setmainfont{Times New Roman} % 英文
\newfontfamily\cjkfont{Noto Serif CJK TC} % 可改為你有的字型,如「標楷體」、「微軟正黑體」
\usepackage{hyperref}
\usepackage{xeCJK}
\xeCJKsetup{CJKmath=true}
\setCJKmainfont{Noto Serif CJK TC} % 同上

\author{B12902036 吳述宇 \and {學號} 涂允貞 \and {學號} 陳塱鋒}{2025 Spring CNS}{}{}{}
\renewcommand{\ccsdesc}[2][]{}
% \authorrunning{}
% \Copyright{}
% \ccsdesc[100]{}
% \category{}
\relatedversion{\href{https://github.com/SY-David/CNS-Final-Project}{GitHub}}
\keywords{CAPTCHA, Machine Learning, OCR, Security Evaluation}
% \supplement{}
% \funding{}
%\acknowledgements{}


\nolinenumbers 
%Editor-only macros:: begin (do not touch as author)%%%%%%%%%%%%%%%%%%%%%%%%%%%%%%%%%%

%%%%%%%%%%%%%%%%%%%%%%%%%%%%%%%%%%%%%%%%%%%%%%%%%%%%%%
\let\hideLIPIcs\relax % 確保不衝突
\begin{document}

\maketitle

%TODO mandatory: add short abstract of the document
\begin{abstract}
CAPTCHA is an essential security mechanism widely used to differentiate humans from automated programs. However, recent advancements in deep learning have significantly weakened the robustness of traditional text-based CAPTCHAs. In this work, we propose a comprehensive CAPTCHA evaluation framework that enables customizable generation, configurable visual perturbations, and real-time model-based testing. Our platform combines a user-friendly GUI CAPTCHA generator with adjustable defense parameters and integrates inference capabilities from three distinct models: a lightweight Char-CNN, a moderately complex VGG16 classifier, and the widely-used Tesseract OCR engine. Experimental evaluations conducted across five difficulty levels—from clean to highly distorted—indicate that all tested models maintain at least 60\% accuracy even under severe perturbations, demonstrating moderate robustness. This prototype provides a practical and versatile solution for systematically assessing CAPTCHA designs against modern machine-learning-based threats with human-friendliness.
\end{abstract}

\section{Introduction}
\label{sec:typesetting-summary}

CAPTCHAs (Completely Automated Public Turing tests to tell Computers and Humans Apart) have become a fundamental security measure widely adopted to prevent automated abuse in web applications, such as spam, credential stuffing, and web scraping by bots. Due to their cost-effectiveness and ease of deployment, CAPTCHAs are ubiquitously utilized across numerous online platforms. However, advancements in machine learning, particularly optical character recognition (OCR) and convolutional neural networks (CNNs), have exposed significant vulnerabilities in traditional text-based CAPTCHAs, undermining their effectiveness as a reliable security measure. In response to this emerging threat, our study addresses the following research question -- How can we design and evaluate CAPTCHA systems that are resistant to deep learning-based attacks while maintaining usability for human users? To investigate this issue, we propose a modular CAPTCHA evaluation platform designed to systematically generate customizable CAPTCHA challenges, apply configurable visual perturbations, and conduct real-time testing using machine-learning-based attack models. Our approach involves integrating a user-friendly graphical user interface (GUI) with adjustable security parameters, and employing three representative models of varying complexity for evaluation: a lightweight Char-CNN classifier, a more robust VGG16 model, and the industry-standard Tesseract OCR engine. Unlike previous works that primarily assess CAPTCHA robustness through fixed datasets and limited scenarios, our framework supports extensive systematic testing across multiple configurable difficulty presets. This feature enables a more thorough and granular analysis of CAPTCHA robustness and weaknesses under diverse conditions, surpassing traditional evaluation methodologies that lack flexibility and comprehensive testing capabilities. Our main contributions in this paper include developing a GUI-based CAPTCHA generation platform supporting extensive customization of perturbations and visual styles, integrating three representative attack models (Char-CNN, VGG16, Tesseract OCR) to benchmark CAPTCHA security systematically, providing a unified and modular evaluation framework that enables fine-grained analysis of CAPTCHA effectiveness under controlled perturbation scenarios, delivering empirical results highlighting model robustness, vulnerabilities, and failure modes across different CAPTCHA difficulty tiers. Through this comprehensive framework, we aim to facilitate transparent, reproducible, and detailed evaluations of CAPTCHA designs in the context of evolving AI-driven threats.
\section{Problem Definition}

Lorem ipsum dolor sit amet, consectetur adipiscing elit \cite{DBLP:journals/cacm/Knuth74}. Praesent convallis orci arcu, eu mollis dolor. Aliquam eleifend suscipit lacinia. Maecenas quam mi, porta ut lacinia sed, convallis ac dui. Lorem ipsum dolor sit amet, consectetur adipiscing elit. Suspendisse potenti. Donec eget odio et magna ullamcorper vehicula ut vitae libero. Maecenas lectus nulla, auctor nec varius ac, ultricies et turpis. Pellentesque id ante erat. In hac habitasse platea dictumst. Curabitur a scelerisque odio. Pellentesque elit risus, posuere quis elementum at, pellentesque ut diam. Quisque aliquam libero id mi imperdiet quis convallis turpis eleifend. 

\begin{lemma}[Lorem ipsum]
\label{lemma:lorem}
Vestibulum sodales dolor et dui cursus iaculis. Nullam ullamcorper purus vel turpis lobortis eu tempus lorem semper. Proin facilisis gravida rutrum. Etiam sed sollicitudin lorem. Proin pellentesque risus at elit hendrerit pharetra. Integer at turpis varius libero rhoncus fermentum vitae vitae metus.
\end{lemma}

\begin{proof}
Cras purus lorem, pulvinar et fermentum sagittis, suscipit quis magna.


\proofsubparagraph*{Just some paragraph within the proof.}
Nam liber tempor cum soluta nobis eleifend option congue nihil imperdiet doming id quod mazim placerat facer possim assum. Lorem ipsum dolor sit amet, consectetuer adipiscing elit, sed diam nonummy nibh euismod tincidunt ut laoreet dolore magna aliquam erat volutpat.
\begin{claim}
content...
\end{claim}
\begin{claimproof}
content...
    \begin{enumerate}
        \item abc abc abc \claimqedhere{}
    \end{enumerate}
\end{claimproof}

\end{proof}

\begin{corollary}[Curabitur pulvinar, \cite{DBLP:books/mk/GrayR93}]
\label{lemma:curabitur}
Nam liber tempor cum soluta nobis eleifend option congue nihil imperdiet doming id quod mazim placerat facer possim assum. Lorem ipsum dolor sit amet, consectetuer adipiscing elit, sed diam nonummy nibh euismod tincidunt ut laoreet dolore magna aliquam erat volutpat.
\end{corollary}

\begin{proposition}\label{prop1}
This is a proposition
\end{proposition}

\autoref{prop1} and \cref{prop1} \ldots



\begin{itemize}
\item Ut vitae diam augue. 
\item Integer lacus ante, pellentesque sed sollicitudin et, pulvinar adipiscing sem. 
\item Maecenas facilisis, leo quis tincidunt egestas, magna ipsum condimentum orci, vitae facilisis nibh turpis et elit. 
\end{itemize}

\begin{remark}
content...
\end{remark}

\section{Results}
\subsection{System Design}
\subsection{Experiment Settings}
\subsection{Results and Evaluation}

Nec urna malesuada sollicitudin. Nulla facilisi. Vivamus aliquam tempus ligula eget ornare. Praesent eget magna ut turpis mattis cursus. Aliquam vel condimentum orci. Nunc congue, libero in gravida convallis \cite{DBLP:conf/focs/HopcroftPV75}, orci nibh sodales quam, id egestas felis mi nec nisi. Suspendisse tincidunt, est ac vestibulum posuere, justo odio bibendum urna, rutrum bibendum dolor sem nec tellus. 

\begin{lemma} [Quisque blandit tempus nunc]
Sed interdum nisl pretium non. Mauris sodales consequat risus vel consectetur. Aliquam erat volutpat. Nunc sed sapien ligula. Proin faucibus sapien luctus nisl feugiat convallis faucibus elit cursus. Nunc vestibulum nunc ac massa pretium pharetra. Nulla facilisis turpis id augue venenatis blandit. Cum sociis natoque penatibus et magnis dis parturient montes, nascetur ridiculus mus.
\end{lemma}

Fusce eu leo nisi. Cras eget orci neque, eleifend dapibus felis. Duis et leo dui. Nam vulputate, velit et laoreet porttitor, quam arcu facilisis dui, sed malesuada risus massa sit amet neque.

\section{Related Work}

Morbi eros magna, vestibulum non posuere non, porta eu quam. Maecenas vitae orci risus, eget imperdiet mauris. Donec massa mauris, pellentesque vel lobortis eu, molestie ac turpis. Sed condimentum convallis dolor, a dignissim est ultrices eu. Donec consectetur volutpat eros, et ornare dui ultricies id. Vivamus eu augue eget dolor euismod ultrices et sit amet nisi. Vivamus malesuada leo ac leo ullamcorper tempor. Donec justo mi, tempor vitae aliquet non, faucibus eu lacus. Donec dictum gravida neque, non porta turpis imperdiet eget. Curabitur quis euismod ligula. 


%%
%% Bibliography
%%

%% Please use bibtex, 

\bibliography{lipics-v2021-sample-article}

\appendix

\section{Conclusion and Future Work}\label{sec:itemStyles}

List of different predefined enumeration styles:

\begin{itemize}
\item \verb|\begin{itemize}...\end{itemize}|
\item \dots
\item \dots
%\item \dots
\end{itemize}

\begin{enumerate}
\item \verb|\begin{enumerate}...\end{enumerate}|
\item \dots
\item \dots
%\item \dots
\end{enumerate}

\begin{alphaenumerate}
\item \verb|\begin{alphaenumerate}...\end{alphaenumerate}|
\item \dots
\item \dots
%\item \dots
\end{alphaenumerate}

\begin{romanenumerate}
\item \verb|\begin{romanenumerate}...\end{romanenumerate}|
\item \dots
\item \dots
%\item \dots
\end{romanenumerate}

\begin{bracketenumerate}
\item \verb|\begin{bracketenumerate}...\end{bracketenumerate}|
\item \dots
\item \dots
%\item \dots
\end{bracketenumerate}

\begin{description}
\item[Description 1] \verb|\begin{description} \item[Description 1]  ...\end{description}|
\item[Description 2] Fusce eu leo nisi. Cras eget orci neque, eleifend dapibus felis. Duis et leo dui. Nam vulputate, velit et laoreet porttitor, quam arcu facilisis dui, sed malesuada risus massa sit amet neque.
\item[Description 3]  \dots
%\item \dots
\end{description}

\cref{testenv-proposition} and \autoref{testenv-proposition} ...

\section{References}\label{sec:theorem-environments}

List of different predefined enumeration styles:

\begin{theorem}\label{testenv-theorem}
Fusce eu leo nisi. Cras eget orci neque, eleifend dapibus felis. Duis et leo dui. Nam vulputate, velit et laoreet porttitor, quam arcu facilisis dui, sed malesuada risus massa sit amet neque.
\end{theorem}

\begin{lemma}\label{testenv-lemma}
Fusce eu leo nisi. Cras eget orci neque, eleifend dapibus felis. Duis et leo dui. Nam vulputate, velit et laoreet porttitor, quam arcu facilisis dui, sed malesuada risus massa sit amet neque.
\end{lemma}

\begin{corollary}\label{testenv-corollary}
Fusce eu leo nisi. Cras eget orci neque, eleifend dapibus felis. Duis et leo dui. Nam vulputate, velit et laoreet porttitor, quam arcu facilisis dui, sed malesuada risus massa sit amet neque.
\end{corollary}

\begin{proposition}\label{testenv-proposition}
Fusce eu leo nisi. Cras eget orci neque, eleifend dapibus felis. Duis et leo dui. Nam vulputate, velit et laoreet porttitor, quam arcu facilisis dui, sed malesuada risus massa sit amet neque.
\end{proposition}

\begin{conjecture}\label{testenv-conjecture}
Fusce eu leo nisi. Cras eget orci neque, eleifend dapibus felis. Duis et leo dui. Nam vulputate, velit et laoreet porttitor, quam arcu facilisis dui, sed malesuada risus massa sit amet neque.
\end{conjecture}

\begin{observation}\label{testenv-observation}
Fusce eu leo nisi. Cras eget orci neque, eleifend dapibus felis. Duis et leo dui. Nam vulputate, velit et laoreet porttitor, quam arcu facilisis dui, sed malesuada risus massa sit amet neque.
\end{observation}

\begin{exercise}\label{testenv-exercise}
Fusce eu leo nisi. Cras eget orci neque, eleifend dapibus felis. Duis et leo dui. Nam vulputate, velit et laoreet porttitor, quam arcu facilisis dui, sed malesuada risus massa sit amet neque.
\end{exercise}

\begin{definition}\label{testenv-definition}
Fusce eu leo nisi. Cras eget orci neque, eleifend dapibus felis. Duis et leo dui. Nam vulputate, velit et laoreet porttitor, quam arcu facilisis dui, sed malesuada risus massa sit amet neque.
\end{definition}

\begin{example}\label{testenv-example}
Fusce eu leo nisi. Cras eget orci neque, eleifend dapibus felis. Duis et leo dui. Nam vulputate, velit et laoreet porttitor, quam arcu facilisis dui, sed malesuada risus massa sit amet neque.
\end{example}

\begin{note}\label{testenv-note}
Fusce eu leo nisi. Cras eget orci neque, eleifend dapibus felis. Duis et leo dui. Nam vulputate, velit et laoreet porttitor, quam arcu facilisis dui, sed malesuada risus massa sit amet neque.
\end{note}

\begin{note*}
Fusce eu leo nisi. Cras eget orci neque, eleifend dapibus felis. Duis et leo dui. Nam vulputate, velit et laoreet porttitor, quam arcu facilisis dui, sed malesuada risus massa sit amet neque.
\end{note*}

\begin{remark}\label{testenv-remark}
Fusce eu leo nisi. Cras eget orci neque, eleifend dapibus felis. Duis et leo dui. Nam vulputate, velit et laoreet porttitor, quam arcu facilisis dui, sed malesuada risus massa sit amet neque.
\end{remark}

\begin{remark*}
Fusce eu leo nisi. Cras eget orci neque, eleifend dapibus felis. Duis et leo dui. Nam vulputate, velit et laoreet porttitor, quam arcu facilisis dui, sed malesuada risus massa sit amet neque.
\end{remark*}

\begin{claim}\label{testenv-claim}
Fusce eu leo nisi. Cras eget orci neque, eleifend dapibus felis. Duis et leo dui. Nam vulputate, velit et laoreet porttitor, quam arcu facilisis dui, sed malesuada risus massa sit amet neque.
\end{claim}

\begin{claim*}\label{testenv-claim2}
Fusce eu leo nisi. Cras eget orci neque, eleifend dapibus felis. Duis et leo dui. Nam vulputate, velit et laoreet porttitor, quam arcu facilisis dui, sed malesuada risus massa sit amet neque.
\end{claim*}

\begin{proof}
Fusce eu leo nisi. Cras eget orci neque, eleifend dapibus felis. Duis et leo dui. Nam vulputate, velit et laoreet porttitor, quam arcu facilisis dui, sed malesuada risus massa sit amet neque.
\end{proof}

\begin{claimproof}
Fusce eu leo nisi. Cras eget orci neque, eleifend dapibus felis. Duis et leo dui. Nam vulputate, velit et laoreet porttitor, quam arcu facilisis dui, sed malesuada risus massa sit amet neque.
\end{claimproof}

\end{document}
